
\section{Impact of scenarios on a reinsurance company solvency ratios}

put everything together on realistic scenarios


The impact of several scenarios on the financial structure and the required capital of a artificial reinsurance company is studied. The impact of the financial statement was presented in section \ref{sec:BS_SHOCK} and on the regulatory required capital in section \ref{sec:RRC_SHOCK}. 

Scenarios proposed are :
\begin{itemize}
\itemsep0em 
\item Natural catastrophes
\item Financial crisis
\item Pandemic
\item A reserve stress test based on natural catastrophes and financial crisis
\end{itemize}





\subsubsection{Description of the company tested}

!!!!!!!!!!!!!!!!!!
nb companies, financial structure, retrocession

The companies belonging to the Group are :

Group in Bermuda
P and C in BERmuda
P and C in Us 
Life and Health in US



\import{sections/section1/scenarios/}{nat_cat.tex}

\import{sections/section1/scenarios/}{fin_crisis.tex}

\import{sections/section1/scenarios/}{pandemic.tex}


\subsubsection{Reverse stress test}

A reverse stress test if a stress test that leads to business failure. A definition of "business failure" must be chosen to calibrate the reverse stress test. Here, several definition are proposed and the reverse stress test is considered a "success" if at least one definition if fulfilled.

\begin{itemize}
\item Group Solvency ratio falls below 100\%
\item At least one Legal Entity solvency ratio falls below 100\%
\item The Group Capital rating falls below BBB
\end{itemize}


The reverse stress test proposed is a combination of natural catastrophes and a financial crisis, whose characteristics are provided below :

\begin{itemize}
\item Europe Windstorm
\item California Earthquake
\item Florida Hurricane
\item Global Financial Crisis
\end{itemize}


