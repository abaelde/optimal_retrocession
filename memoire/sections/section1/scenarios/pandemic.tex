


\subsubsection{Pandemic scenario}

The word pandemic is used to describe the case of the propagation of a disease in the whole planet. Numerous pathologies can be classified as pandemic : influenza, AIDS, smallpox and even obesity. The most important aspect of a pandemic the fact that the disease, whatever it is, is spread everywhere on the planet.

For extreme scenario, only the case of influenza will be considered. The reasons for this choice are :
\begin{itemize}
\item The speed of propagation of this disease : the virus is easily transmitted through respiration or contact, unlike AIDS.
\item The fact that it is liked to a pathogen agent and to not hygiene conditions (like obesity which is linked with food intake and genetic).
\item The fact that new stems of virus are active and make such event probable.
\item The high infection rate but low fatality rate of such disease, which can lead to strong economic disruption but is unlikely to eradicate humanity.
\item The strong literature dealing with this disease.
\end{itemize}

The characteristics of the chosen scenario are :

\begin{itemize}
\item Return period : 1 in 200 years
\item Excess Mortality : 1 \textperthousand\ to 2.5 \textperthousand\ depending on the country
\item Case Mortality (number of infected who die) : 3 \%
\item Number of hospitalized people over 1 death : 4
\item W-shaped mortality curve in function of people age : specific to pandemic contrary to classic epidemic
\end{itemize}

It is assumed that the pandemic is worldwide and last 3 month.

Mortality rates are computed using the Swiss Re pandemic model allowing to compute mortality rates per countries and age of a pandemic close to the 2019 Spanish Flu pandemic.

Such a pandemic might have strong economic impacts. Schools will probably be close to prevent to propagation of the virus, leading to large absenteeism at work. Companies being crippled by this shortage of workers might reduce their activities and maybe go bankrupt. A decrease of demand might be experienced in specific sectors involving face to face contact like tourism, entertainment, or the gathering of people in the same place, like airlines, public transportations and shopping malls. All these elements might lead to a global recession.

The impact of such pandemic on several Life and Non-Life business lines is presented below. Naturally the impact of a pandemic on Non-Life business lines is difficult to estimate. Nevertheless, the reader will at least find a list of potentially impacted business lines and the drivers behind the impact.


\paragraph{Mortality}

The Mortality business line is the most obvious business line impacted by a pandemic. As the pandemic develops, people dies, which trigger death policy. A simple way of computing the impact of a mortality increase is to multiply the sum at risk, which is the sum that should be paid if all policyholders dies, with the excess mortality, which is the additional fraction of people dying because of the pandemic. This sum at risk can be spitted by country and age to refine the calculation.

\paragraph{Longevity}

As more people die because of the pandemic, less pensions are paid. This phenomenon hedges to mortality loss but full calculations of this effect are out of the scope of this work. No hedging will be considered in this work for sake of conservatism.

\paragraph{Health}

As people hospitalized, the cost of care might increases dramatically. But humans can only receive a limited quantity of care before dying. Moreover, hospitals might be saturated by the number of patient, therefore limiting the final amount of heathcare costs. Thus, there is a non linear effect where the total amount of healthcare cost saturate at a limit which is the limit of the healthcare system.


\paragraph{Credit \& Surety}

As people try to avoid shopping malls during a pandemic, large retailers might experience losses and difficulties to pay their suppliers. These suppliers being insured by trade credit insurers against default of payment, it might lead to loss to reinsurers reinsuring trade credit insurers. a method to compute the impact of a pandemic on a trade credit portfolio is to use the 2008 financial crisis a reference case. It consists in comparing the loss ratio of the 2008 or 2009 year to the average of "normal" years and then identifying the contribution of the financial crisis in terms of additional loss ratio. This contribution is then applied on the current portfolio. Even if this method is indirect, it can be considered as conservative as a pandemic economic disruption might be less severe the the 2008 financial crisis.

\paragraph{Business Interruption}

The shortage or supply and demand might lead some businesses to stop activity, and then trigger business interruption covers. Estimating the impact of the pandemic on business interruption of difficult as cover triggers can be very different from one portfolio to another. One key aspect is that most business interruption cover are triggered by physical damage to the "company" (building etc). The pandemic is not considerer as a "physical damage" which leads us to think that most covers would not be triggered.


\paragraph{Marine}

During a pandemic, harbors might forbid ships to dock or leave, fearing to bear the responsibility of the propagation of the virus. Loss to reinsurer could arise if perishable cargo is destroyed for example.

\paragraph{Sport, Leisure \& Entertainments}

A pandemic could lead top the canceling of world cup or concerts.
