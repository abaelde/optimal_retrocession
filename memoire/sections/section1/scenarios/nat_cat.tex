\subsubsection{Natural catastrophe scenario}

Most reinsurers underwrite risk associated with natural catastrophes. The occurrence of natural catastrophes (hurricanes, earthquakes, floods, wildfires, etc) poses huge problems for insurer as a large part of their portfolio can be hit by such event. This is due to their local implantation that prevents them to diversify risk on a worldwide basis, and their reluctance to bear such extreme risk.
In the context of stress testing, an reinsurer must know its exposure by type of peril and peril zone. This is generally done by dedicated catastrophe modeling teams, relying on catastrophe modeling software such as RMS or AIR. A lot of reinsurer were created after years marked with strong natural catastrophes, such as hurricane Andrews, which led to the creation of PartnerRe.

The probability distribution of losses per peril type and zone is computed by simulating the impact of hundred of thousand of catastrophe scenarios on the company portfolio. The gross loss of a natural catastrophe is associated with an occurrence probability ie. a return period : the higher the return period, the higher the severity of the event. This information is useful when designing scenarios : with proper catastrophe modeling, the loss applied on the company is chosen according to a desired return period. Some scenarios might involve whether numerous small catastrophes or few very severe catastrophes. The knowledge of returns period allow to apply shocks whose likelihood is known.



The scenario chosen is the case of a hurricane hitting the Florida coast and an earthquake arising in California. The loss distribution of these events are displayed in :


\begin{table}
\centering
\begin{tabular}{|l|l|l|}
\hline
   \textbf{Return Period} & \textbf{Florida Hurricane} & \textbf{California Earthquake} \\ \hline \hline
   1-500 & 0 & 0 \\ \hline
   \label{t:CAT_NAT_EVENT}
\end{tabular}
   \caption{Events defined in the Natural catastrophes scenario.}
\end{table}

In order for this scenario to have a meaningful impact, a return period of 1 in 250 years is chosen. As these two events can be considered as independent, their respective return period can be chosen as 1 in 15 years.

The gross losses of these two events are : 1-15 y GROSS LOSS


After usage of the company external retrocession treaties, the net loss are : NET LOSSES

The net loss is then spread through the company's structure through the use of its internal retrocession agreements.

This lead to the following repartition of losses in the legal entities :

\begin{table}
\centering
\begin{tabular}{|l|l|l|}
\hline
   \textbf{Legal entity} & \textbf{1} & \textbf{2} \\ \hline \hline
   Net Loss (of external retrocession) & 0 & 0 \\ \hline
   Net Loss (of external and internal retrocession) & 0 & 0 \\ \hline
   \label{t:CAT_NAT_LOSS}
\end{tabular}
   \caption{Losses bore by legal entities, net of external retrocessions and net of external and internal retrocessions.}
\end{table}


The required capital are modified through the mechanisms explained in section \ref{sec:RRC_SHOCK}, leading to the following values :

\begin{table}
\centering
\begin{tabular}{|l|l|l|}
\hline
   \textbf{Legal entity} & \textbf{Group} & \textbf{1} \\ \hline \hline
   Initial Required Capital & 0 & 0 \\ \hline
   Final Required Capital & 0 & 0 \\ \hline
   \label{t:CAT_NAT_RRC}
\end{tabular}
   \caption{Evolution of the required capital of Group and legal entities before and after the scenario impact.}
\end{table}


Using informations from table\ref{t:CAT_NAT_LOSS} and table \ref{t:CAT_NAT_RRC}, the solvency ratio of Group and legal entities is displayed in table \ref{t:CAT_NAT_SR} :

\begin{table}
\centering
\begin{tabular}{|l|l|l|}
\hline
   \textbf{Legal entity} & \textbf{Group} & \textbf{1} \\ \hline \hline
   Initial Solvency Ratio & 0 & 0 \\ \hline
   Final Solvency Ratio & 0 & 0 \\ \hline
   \label{t:CAT_NAT_SR}
\end{tabular}
   \caption{Evolution of Group and legal entities solvency ratio before and after the scenario impact.}
\end{table}


GRAPHE PERTES