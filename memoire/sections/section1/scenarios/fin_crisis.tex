
\subsubsection{Financial crisis scenario}

A financial crisis is a very dangerous event for a reinsurance company. In fact, contrary to services or industrial companies who will suffer from a decrease of activity in the future, a reinsurance company will also have difficulties to pay its past liabilities. The financial crisis may decrease the value of the company's assets : there is less money to pay claims of contracts signed previously, leading to an increased risk of insolvency. 

A number of financial crisis arose in the XXth and XXIth century : EXAMPLE CRISIS


Their characteristics are always different and they are very difficult to forecast. 


A financial crisis has an impact on asset value, which can be considered as a loss, and also on the regulatory required capital, as asset risk are generally taken into account. Methods to shock assets are presented in section \ref{sec:BS_SHOCK}. The time aspect is key to understand the scenario impact. In fact, even if the asset value decreases a lot at the peak of the crisis, they might stabilize at a value which is not too far from their original value. But financial statements and the computation of regulatory required capital are done at a given date. The worst case is when the peak of the crisis happen at the closing date, which is the case in the scenarios considered here.


PROVIDE VALUE WITH JUSTIFICATION


The losses associated with such crisis are  the following :

\begin{table}
\centering
\begin{tabular}{|l|l|l|}
\hline
   \textbf{Legal entity} & \textbf{Group} & \textbf{1} \\ \hline \hline
   Fixed income losses & 0 & 0 \\ \hline
   Equity Losses & 0 & 0 \\ \hline
   Other investments losses & 0 & 0 \\ \hline
   \label{t:FC_LOSS}
\end{tabular}
   \caption{Evolution of Group and legal entities assets before and after the scenario impact.}
\end{table}

As no risk transfer or mitigation mechanism exists in the case of asset losses, the asset allocation must be done carefully to ensure the solvency of all legal entities.

Solvency ratios before and after the financial crisis are displayed in :


\begin{table}
\centering
\begin{tabular}{|l|l|l|}
\hline
   \textbf{Legal entity} & \textbf{Group} & \textbf{1} \\ \hline \hline
   Initial Solvency Ratio & 0 & 0 \\ \hline
   Final Solvency Ratio & 0 & 0 \\ \hline
   \label{t:FC_SR}
\end{tabular}
   \caption{Evolution of Group and legal entities solvency ratio before and after the financial crisis scenario.}
\end{table}




Some impacts are neglected but could be taken into account in other work. For example, a financial crisis would probably reduce the premium volume collected. This can be the case for aviation insurance where the premium volume is proportional to the air traffic. The bankruptcy of a big customer can also lead to a loss of premium income.
